\section{Аналитический раздел}
В аналитическом разделе опишем основные сущности базы данных, приведем диаграммы
IDEF0, DFD и ER.

\subsection{Постановка задачи}
Целью данного курсового проекта является реализация информационной системы, состоящей из
базы данных и пользовательского приложения. Для достижения данной цели необходимо заранее
спроектировать схему базы данных. Для достижения наилучшего результата необходимо выделить
набор сущностей и связи между ними. Далее строится ER\,--\,диаграмма и создаются необходимые
таблицы. Для автоматизирования этого процесса была использована система MySQL~Workbench.
\newpar
Помимо самой базы данных необходимо написать графическое приложение в виде веб-сайта,
которое предоставит доступ к данным в базе данных, а так же позволит редактировать и дополнять
базу данных.

\subsection{Сущности предметной области}
Приведем набор сущностей, исходя из которых была спроектирована база данных:
\begin{description}
    \item[Человек] характеризуется набором необходимых параметров, таких как имя и
        фамилия. Комбинация имя/фамилия привязывается к адресу электронной почты человека,
        чем гарантируется возможность добавления нескольких человек с одной фамилией и
        именем. Система не допускает существование двух одинаковых сущностей <<Человек>> с
        одним адресом электронной почты.
    \item[Соревнование] характеризуется годом в который оно проводится. Не допускается
        участие одного человека дважды в один год.
    \item[Роль] характеризует позицию человека в системе. В зависимости от года, когда
        проходит соревнование, человек может иметь разные роли. Для конкретного года
        соревнования человек может иметь только одну роль.
    \item[Учебное заведение] характеризуется своим номером и/или названием и типом, является местом,
        которое представляет участник во время соревнования. Допускается участие
        нескольких участников из одной школы в один год в одном соревновании.
    \item[Работа] является объектом оценки достижений человека в определенный год
        соревнования. Работа принадлежит определенному участнику из определенной школы.
        Работа имеет куратора. Работа может принадлежать только одному участнику.
    \item[Критерий] характеризует параметр по которому происходит оценка работы участника в
        определенный год соревнования. Допускается множество критериев для оценки. Набор
        критериев сохраняется из года в год.
\end{description}

Приведем список ролей, которые может иметь человек:
\begin{itemize}
    \item Эксперт \\
        Провоизводит оценку работ, выставленных на соревновании
    \item Рецензент \\
        Составляет рецензии на выставленные на соревновании работы
    \item Администратор \\
        Осуществляет административные функции (например добавление участников в соревнование,
                распределение кураторов и тп.)
    \item Куратор \\
        Занимается курированием работы участника соревнования
    \item Участник \\
        Выставляет свою работу на соревнование
\end{itemize}

Заметим, что в приведенном списке присутствуют лишь основные сущности базы данных, без
уточнения порожденных в результате нормализации дополнительных сущностей.

\subsection{Проектирование базы данных}
Под атрибутами в ER\,--\,модели понимаются некоторые значимые характеристики сущности.
\newpar
На основе приведенных выше сущностей, далее определим для них набор параметров и
атрибутов:
\begin{itemize}
    \item Человек
        \begin{itemize}
            \item Уникальный идентификационный номер
            \item Имя
            \item Фамилия
            \item Отчество
            \item Пол
            \item Адрес электронной почты
            \item Дата рождения
            \item Формальное описание
            \item Адрес
            \item Номер телефона
        \end{itemize}
    \item Соревнование
        \begin{itemize}
            \item Уникальный идентификационный номер
            \item Год соревнования
        \end{itemize}
    \item Роль
        \begin{itemize}
            \item Уникальный идентификацонный номер
            \item Идентификационный номер человека
            \item Идентификационный номер соревнования
            \item Конкретная роль
        \end{itemize}
    \item Учебное заведение
        \begin{itemize}
            \item Уникальный идентификационный номер
            \item Название учебного заведения
            \item Номер учебного заведения
            \item Адрес учебного заведения
            \item Город, где находится учебное заведение
            \item Тип учебного заведения, например школа, гимназия, лицей и тп.
        \end{itemize}
    \item Работа
        \begin{itemize}
            \item Уникальный идентификационный номер
            \item Идентификационный номер участника соревнования
            \item Идентификационный номер учебного заведения
            \item Идентификационный номер куратора соревнования
            \item Статус работы
            \item Название работы
            \item Дата регистрации работы
        \end{itemize}
        Статус работы может быть следующим:
        \begin{itemize}
            \item Статус неизвестен
            \item Работа прошла предварительный отбор
            \item Работа не прошла предварительный отбор
        \end{itemize}
    \item Критерий
        \begin{itemize}
            \item Уникальный идентификационный номер
            \item Название критерия
            \item Полное название критерия
        \end{itemize}
\end{itemize}

Заметим, что фактическое количество таблиц в базе данных может быть большим, чем число
сущностей.

\subsection{Нормализация базы данных}
Для того, чтобы предотвратить некоторые аномалии (аномалии вставки, аномалии удаления,
аномалии обновления), которые возникают в процессе проектирования базы данных, обычно
применяют нормализацию. Нормализация обеспечивает систематический подход к распределению
атрибутов по таблицам, тем самым делая базу данных более надежной с точки зрения
непротиворечивости данных. Данная база данных на этапе проектирования была доведена до
третьей нормальной формы. Это было сделано для более простой и более быстрой обработки
данных.
\newpar
Далее рассмотрим нормализацию спроектированной базы данных.
\newpar
Первая нормальная форма предполагает хранение одного значения в одном столбце, что
реализовано при проектировании. Так как каждая таблица содержит уникальный
идентификационный номер, который является первичным ключем, а так же является логическим ключем,
то есть, является минимальным набором атрибутов, то база данных находится во второй
нормальной форме. В свою очередь, благодаря отсутствию транзитивных зависимостей, база
данных находится в третьей нормальной форме.

\subsection{Ограничения}
Ограничения -- это элементы определения таблицы, ограничивающие значения, вводимые в
таблицу. Определим ограничения, которые возникли при проектировании базы данных:
\begin{itemize}
    \item Первичные ключи необходимы, чтобы однозначно идентифицировать строку в таблице,
        они не могут иметь нулевых значений;
    \item Для связи таблиц друг с другом использубтся внешние ключи, которые так же не могут быть
        нулевыми и должны указывать на некоторую строку во внешней таблице для поддержания
        ссылочной ценности;
\end{itemize}

\subsection{ER\,--\,модель}
Для представления модели используется диаграмное представление, ER\,--\,диаграмма.
\begin{figure}[tb]
    \centerline{\includegraphics[width=1.25\textwidth]{er.eps}}
    \caption{Полная ER\,--\,диаграмма}\label{er}
\end{figure}

На рисунке \ref{er} изображена полная ER\,--\,диаграмма, которая отображает
нормализованную до третьей нормальной формы схему базы данных.
\newpar
Диаграмма построена по следующим правилам:
\begin{itemize}
    \item Сущность изображена в виде прямоугольника, содержащем ее имя.
    \item Связь изображается линией, которая связывает две сущности, участвующие в
        отношении.
    \item Множественность связи изображается в виде <<вилки>> на конце связи.
    \item Атрибуты сущности записываются внутри прямоугольника, изображающего сущность.
    \item Среди атрибутов выделяется ключ сущности -- неизбыточный набор атрибутов,
        значения которых в совокупности являются уникальными для каждого экземпляра
        сущности.
\end{itemize}

\subsection{Диаграмма DFD}
Данная диаграмма характеризует поток информации в системе.
\begin{figure}[htb]
    \centerline{\includegraphics[width=1.25\textwidth]{dfd.eps}}
    \caption{DFD\,--\,диаграмма}\label{dfd}
\end{figure}
Была использована нотация Йордана, в которой:
\begin{itemize}
    \item Потоки изображаются стрелками
    \item Процессы обозначаются окружностями
    \item Накопитель данных, в нашем случае это база данных, обозначается скругленным
        прямоугольником
    \item Внешние сущности обозначаются прямоугольниками
\end{itemize}

\subsection{Авторизация}
Под авторизацией понимается предоставление прав доступа определенному субъекту.
Авторизация необходима для того, чтобы различные пользователи могли выполнять различные
функции в пределах одной системы.
\newpar
Система различает 2 вида авторизованных субъектов:
\begin{itemize}
    \item Администратор может воспользоваться следующими функциями
        \begin{itemize}
            \item Регистрация новых персон
            \item Регистрация аккаунта для зарегестрированной персоны
            \item Добавление учебного заведения в список
            \item Добавление критериев оценки работ участников
            \item Распределение ролей в соревновании
            \item Объявление начала соревнования
        \end{itemize}
    \item Пользователь может воспользоваться следующими функциями
        \begin{itemize}
            \item Просмотр списков работ
            \item Если пользователь в этом году является <<Экспертом>>,
                проставление оценок за работы
        \end{itemize}
\end{itemize}

Для доступа в систему необходимо знание логина и пароля пользователя.
Вне системы, пользователь может просмотреть лишь краткую информацию о проекте.
\newpar
Для хранения данных об авторизованных пользователях используется механизм \textit{сессий}.
При первом удачном входе пользователя в систему, создается уникальный идентификатор
сессии, который используется данным пользователем до выхода из системы. Благодаря
механизму \textit{cookies} пользователю нет необходимости повторять вход даже после
закрытия окна браузера.
