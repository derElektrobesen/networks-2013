\section{Исследвательский раздел}
В этом разделе будет представлено сравнение быстродействия
некоторых операций для СУБД MySQL и системы Memcached.

\subsection{Исследование быстродействия MySQL и Memcached}
Для хранения данных, к которым требуется частый доступ без обработки
используются NoSQL решения. Было решено использовать для хранения
сессий пользователей систему кэширования Memcached. На рисунке \ref{memsql}
показан график времени выполнения операция для Memcached и MySQL.

\begin{figure}[!hbt]
    \centering
    \includegraphics{memsql}
    \caption{Сравнение выполнения операций на MySQL и Memcached}\label{memsql}
\end{figure}

\subsection*{Выводы}
\addcontentsline{toc}{subsection}{Выводы}
В результате проведенной работы был разработан программный
комплекс, состоящий из клиента и сервера, которые могут быть запущены
практически на любой операционной системе.
\newpar
Для взаимодействия между
клиентом и сервером был разработан специальный текстовый протокол,
который удобно отслеживать в сниффере. Разработанный программный
комплекс поддерживает автоматическое обнаружение сервера, авторизацию
пользователей, журналирование, обработку ошибок, имеет современный веб-интерфейс
и полностью удовлетворяет поставленным требованиям.

