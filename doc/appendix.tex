\begin{appendix}
\section*{Приложение А}
\newsection{Приложение А}
\addcontentsline{toc}{section}{Приложение А}
В приложении приводятся основные прототипы пользовательских функций,
структур и некоторых типов данных.
\subsubsection*{Прототипы функций и структуры, используемые сервером}
Структура приведенная ниже описывает характеристики куска файла, такие
позиции начала и конца куска, содержит сам кусок, имя файла и указатель
на объект \texttt{FILE}. Используемые константы:
\begin{itemize}
    \item \texttt{CACHED\_PIECES\_COUNT}~-- число кусков находящихся в кэше.
    \item \texttt{DATA\_BLOCK\_LEN}~-- размер куска файла.
\end{itemize}
\begin{lstlisting}
struct file_cache {
    unsigned char data[CACHED_PIECES_COUNT * DATA_BLOCK_LEN];
    piece_id_t start_piece;
    piece_id_t end_piece;
    char name[DATA_BLOCK_LEN];
    FILE *file;
};
\end{lstlisting}

Структура, которая реализует очередь кусков файлов.
\texttt{MAX\_TRANSMISSIONS}~-- максимальное число передач.
\begin{lstlisting}
struct files_queue {
    char positions[MAX_TRANSMISSIONS];
    struct file_cache cache[MAX_TRANSMISSIONS];
    int count;
};
\end{lstlisting}

Функция, которая обрабатывает сообщение переданное клиентом.\\
\texttt{int process\_client\_message(int sender\_sock, const char *msg, size\_t count);}

Функция ниже используется для обмена сообщениями с пользовательским
приложением.\\
\texttt{void setup\_gui\_acts(struct gui\_actions *acts);}

\subsubsection*{Прототипы функций и структуры, используемые клиентом}
Структура описывает кусок, который выбивается из структуры ниже
\begin{lstlisting}
struct file_udata_t {
    piece_id_t piece_id;    \\ Индекс куска файла
    piece_len_t piece_len;  \\ Текущий размер куска
    unsigned char data[DATA_BLOCK_LEN]; \\ Кусок данных
};
\end{lstlisting}

Структура описывает блок данных для сброса на жесткий диск.
\begin{lstlisting}
struct file_data_t {
    piece_id_t s_piece;
    piece_id_t f_piece;
    int pieces_copied;
    size_t full_size;
    unsigned char data[CACHED_PIECES_COUNT* DATA_BLOCK_LEN];
};
\end{lstlisting}

Структура объединяет структуры выше, чтобы не передавать лишние па-
раметры
\begin{lstlisting}
struct file_full_data_t {
    struct file_data_t data;
    struct file_udata_t udata[CACHED_QUEUE_LEN];
};
\end{lstlisting}

Структура описывает одно активное соединение (подключение)
\begin{lstlisting}
struct active_connection {
    int srv_sock;
    int timeout;
    int transmission_id;
    int status;
    piece_id_t piece_id;
    file_id_t file_id;
    pack_id_t pack_id;
    struct file_full_data_t *data;
    struct active_connection *next;
    struct active_connection *prev;
};
\end{lstlisting}

Список всех активных соединений (подключений)
\begin{lstlisting}
struct active_connections {
    struct active_connection *q_head;
    struct active_connection *q_tail;
};
\end{lstlisting}

Структура для описания части файла, которую необходимо запросить
\begin{lstlisting}
struct pieces_queue {
    piece_id_t max_piece_num; \\ Максимальный номер куска
    piece_id_t cur_piece; \\ Содержит минимальный не запрошенный номер куска
    signed int max_failed_piece_num; \\ Максимальный номер куска в failed_pieces
    piece_id_t failed_pieces[MAX_PIECES_COUNT]; \\ Содержит неполученные куски файла
    piece_id_t flushed_pieces_count; \\ Число записанных на диск кусков
};
\end{lstlisting}

Структура определяет одну активную передачу
\begin{lstlisting}
struct transmission {
    FILE *file;
    char filename[FILE_NAME_MAX_LEN];
    unsigned char filesum[MD5_DIGEST_LENGTH];
    unsigned short status;
    size_t filesize;
    struct pieces_queue pieces;
};
\end{lstlisting}

Структура описывает список всех активных передач
\begin{lstlisting}
struct transmissions {
      struct transmission trm[MAX_TRANSMISSIONS];
      unsigned char openned_trms[MAX_TRANSMISSIONS];
      unsigned int count; \\ Число закрытых передач
};
\end{lstlisting}

Функция обрабатывает сообщения, полученные от сервера\\
\texttt{int process\_srv\_message(int sock, const char *msg, size\_t len);}
\newpar
Функция, реализующая работу диспетчера\\
\texttt{void main\_dispatcher();}
\newpar
Функция, запускающая посылку файла\\
\texttt{int receive\_file(const char *filename, const
unsigned char *hsum, unsigned long fsize, const struct sockets\_queue *q);}
Функция для обмена сообщениями с пользовательским интерфейсом\\
\texttt{void setup\_gui\_acts(struct gui\_actions *acts);}
\subsubsection*{Прототипы функций модуля Network}
Функция, которая запускает процесс-сервер\\
\texttt{int start\_server(socket\_callback process\_cli\_msg\_callback);}
\newpar
Функция, которая запускающая процесс-клиент\\
\texttt{int start\_client(socket\_callback
process\_srv\_msg\_callback,
queue\_dispatcher dispatcher, struct sockets\_queue *q);}
\newpar
Функция, используемая для пересылки данных\\
\texttt{int send\_data(int sock, char *buf, int len, int flags);}
\newpar
Функция, необходимая для обмена сообщениями
с пользовательским приложением\\
\texttt{void setup\_gui\_msgs(struct gui\_actions *acts);}
\subsubsection*{Прототипы функция для обработки сообщений}
Десериализация, полученного сообщения от клиента\\
\texttt{int encode\_cli\_msg(struct cli\_fields *fields, const char *msg, int
msg\_len);}
\newpar
Сериализация сообщения посылаемого клиенту\\
\texttt{int decode\_cli\_msg(const struct cli\_fields *fields, char *msg);}
\newpar
Десериализация, полученного сообщения от сервера\\
\texttt{int encode\_srv\_msg(struct srv\_fields *fields, const char *msg, int
msg\_len);}
\newpar
Сериализация сообщения посылаемого серверу\\
\texttt{int decode\_srv\_msg(const struct srv\_fields *fields, char *msg);}
\newpar
Декодирование ошибок (интерпретация)\\
\texttt{void decode\_proto\_error(perror\_t e, char *s, int max\_len);}

\subsubsection*{Callback-функции для обмена сообщениями с пользовательским приложением}
\begin{lstlisting}
struct gui_actions {
    gui_recv_callback start_trm;
    gui_recv_callback stop_trm;
    gui_recv_callback terminate;
    gui_send_callback answer;
    gui_send_callback package_sent;
    gui_send_callback package_received;
    gui_send_callback server_added;
    gui_send_callback client_added;
    gui_send_callback server_removed;
    gui_send_callback client_removed;
    gui_send_callback file_received;
    int sock;
};
\end{lstlisting}

Опишем смысл функций в порядке следования в структуре:

\begin{enumerate}
    \item начать передачу
    \item остановить передачу
    \item завершить работу
    \item ответить
    \item отправленные пакеты
    \item полученные пакеты
    \item добавление клиента
    \item добавление сервера
    \item удаление клиента
    \item удаление сервера
    \item файл получен
\end{enumerate}

Следующая структура хранит массив активных сокетов
\begin{lstlisting}
struct sockets_queue {
    int sockets[MAX_CONNECTIONS];
    in_addr_t addrs[MAX_CONNECTIONS];
    int count;
};
\end{lstlisting}

Выше были приведены основные прототипы и структуры данных, которые не
были указаны в технологическом разделе.

\section*{Приложение Б}
\newsection{Приложение Б}
\addcontentsline{toc}{section}{Приложение Б}
В данном приложении будут представлены исходные коды некоторых программных модулей проекта.
\subsection*{Исходные коды модуля \texttt{proto.c}}
\addcontentsline{toc}{subsection}{Исходные коды модуля \texttt{proto.c}}
\lstinputlisting[language=C]{../backend/proto.c}
\subsection*{Исходные коды модуля \texttt{srv.c}}
\addcontentsline{toc}{subsection}{Исходные коды модуля \texttt{srv.c}}
\newsection{Приложение Б}
\lstinputlisting[language=C]{../backend/srv.c}
\end{appendix}
