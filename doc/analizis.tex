\section{Исследователькая часть}
\newsection{Исследовательская часть}
В этой части, в главе \ref{exists}, приведем примеры существующих решений,
опишем их плюсы и минусы. В главе \ref{proto} кратко опишем разрабатываемое
программное обеспечение, так же опишем недостатки и преимущества.

\subsection{Краткое описание существующих решений}\label{exists}
Наиболее известной одноранговой сетью является \textit{BitTorrent}\cite{bit_torrent},
протокол для которой был разработан в июле 2001 года Брэмом Коэном.
\newpar
Данная сеть позволяет скачивать файл по частям, используя
для этого других участников сети. Участники сети могут запрашивать файлы
для скачивания, а также их одновременно раздавать. Скорость скачивания
зависит от рейтинга каждого участника, рейтинг тем выше, чем больший
объем данных участник раздал. Поиск участников производится при помощи
специального сервера, который хранит данные каждого из участников.
\newpar
Поэтому при выходе из строя такого сервера передача становится
невозможной, однако на сегодняшний день разработан вариант работы
данной сети, не использующей специализированный сервер, то есть система
децентрализована. Также данная система обладает рядом преимуществ,
например режим end game, который позволяет скачать оставшийся
небольшой объем данных более эффективно, что не тормозит работу, если
скорость приема от некоторых серверов низкая.
\newpar
Однако данная сеть обладает рядом недостатков. Если раздача не пользуется
популярностью, то может возникнуть ситуация, когда скачивание файла
становится невозможной. Так как данная сеть может обращаться к другим
участникам, использую глобальную сеть, то другим участникам необходимо
знать IP – адреса других участников, которые они получают от сервера. То
есть возникает проблема анонимности. Поэтому использование данной сети,
например, в организациях, где необходимо обмениваться файлами локально
может быть небезопасно. К недостаткам можно отнести и то, что для
скачивания некоторых файлов необходимо регистрироваться.

\subsection{Описание разработанного протокола}\label{proto}
Программное обеспечение ориентировано на работу в локальной сети, что
является его недостатком. Если в сети имеется лишь один пир, то
использование данного программного обеспечения не имеет смысла, так как
нет серверов, которые могут содержать необходимый файл для скачивания.
\newpar
Так как данный протокол ориентирован на использование в локальной сети,
то передача файлов становится более надежной. Также нет необходимости
вводить централизованный сервер, так как каждый участник, подключившись
к сети, становится участником раздачи (алгоритм работы сети и поведение
участников описаны в конструкторской части). То есть сеть можно считать
децентрализованной, поэтому выход из сети одного из участников не влечет
за собой отмену передачи всего файла, если в сети присутствуют другие
участники имеющие необходимый файл. Отсутствие различных рейтингов
позволяет передавать файл с одинаковой скоростью всем участникам. К
достоинствам также можно отнести то, что графический интерфейс не
привязан к остальной части программного обеспечения и может быть
изменен самим пользователем, если у пользователя имеются необходимые
навыки.

\subsection*{Выводы}
\addcontentsline{toc}{subsection}{Выводы}
При реализации сети на основе пиринговой технологии необходимо
учитывать разрабатывается ли она с учетом работы в глобальной сети или
нет. Если да, то необходимо тщательно подходить к вопросам обеспечения
безопасности и анонимности участников. Так, в рассмотренной выше сети
BitTorrent участники знают адреса других участников. В локальной сети это
может и не иметь большого значения, однако в глобальной сети это является
большим недостатком. Данный вопрос может быть решен путем
децентрализации сети. Тогда, каждый из участников сможет создавать список
тех участников, кому он доверяет. В рамках данной курсовой работы была
реализована децентрализованная сеть.
