\section*{Введение}
\newsection{Введение}
\addcontentsline{toc}{section}{Введение}
йСетевые технологии сегодня очень распространены. Множество
вычислительных систем взаимодействуют друг с другом для выполнения
некоторых общих задач посредством сети. Часто возникает необходимость
обмена данными между вычислительными системами. Одно из решений
обмена данными между различными вычислительными устройствами в
локальной сети является организация системы на основе пиринговой
технологии.
\newpar
Данные в вычислительных системах хранятся в виде файлов, то есть блоков
байтов и могут иметь различные размеры. При обмене данными, данные
обычно передаются определенными кусками, каждый из которых может
иметь различную длину и некоторые другие характеристики. Для того чтобы
обмен данными выполнялся корректно необходимо также передавать
некоторую информацию о передаваемых данных. Для решения данной задачи
необходимо разработать высокоуровневый протокол, который позволит
корректно обмениваться данными и контролировать их целостность.
\newpar
В исследовательской части будут рассмотрены существующие решения
данной задачи.
\newpar
В конструкторской части будет описана структура программного
обеспечения. Также будут приведены и рассмотрены основные алгоритмы.
\newpar
В технологической части будет обоснован выбор языка и средств
программирования, а также выбор операционной системы, для которой
разрабатывалось программное обеспечение.

\section*{Определения}
\newsection{Определения}
\addcontentsline{toc}{section}{Определения}

В настоящем отчете применяются следующие термины и определения:
\begin{description}
    \item[Одноранговая сеть]--- оверлейная сеть, в которой все участники имеют
        одинаковые права.
    \item[Оверлейная сеть]--- сеть, создаваемая поверх другой сети.
    \item[Участник сети]--- приложение, установленное на хосте. Состоит из клиента,
        сервера и пользовательского приложения. Под участником можно понимать
        пользователя, который использует данное приложение, однако далее это будет
        подразумевать именно само приложение.
    \item[Пир]--- участник одноранговой сети, каждый пир является клиентом, а также
        выполняет функции сервера. Данный термин является синонимом участника
        сети и приведен, так как данный термин используется в системе BitTorrent.
    \item[Клиент]--- пир, который запрашивает у других участников (серверов) файлы.
        Под клиентом также будет подразумеваться процесс, который выполняет
        функции клиента.
    \item[Сервер]--- пир, который при запросе от клиента передает запрашиваемый
        файл, если он имеется у сервера. Под сервером также будет подразумеваться
        процесс, который выполняет функции сервера.
    \item[Кусок файла]--- часть файла (последовательность байтов), которую запросил
        клиент у сервера.
    \item[Идентификатор файла]--- целое число, которое указывает номер файла, для
        идентификации передачи.
    \item[Идентификатор куска файла]--- целое число, которое указывает порядковый
        номер куска файла (идентификаторы куска и файла однозначно
        идентифицируют одну передачу). Идентификаторы назначаются клиентом в
        зависимости от количества серверов.
    \item[Список пассивных подключений] (соединений)~--- список, в котором
        содержатся участники сети, которым можно будет послать запрос.
    \item[Список активных подключений] (соединений)~--- список, который содержит
        участников сети, которые на данный момент времени обмениваются
        данными.
    \item[Торрент файл]--- файл определенной структуры, однозначно идентифицирующий возможную передачу.
    \item[Уникальная передача]--- передача, которая является уникальной для клиента,
        то есть некоторый пакет, состоящий из куска, который принадлежит
        передаваемому файлу. Однозначно определяется идентификатором пакета,
        идентификатором файла и идентификатором куска.
    \item[GUI или пользовательское приложение]--- приложение с помощью которого
        пользователь выполняет определенные операции
    \item[Список отвергнутых кусков]--- список, в который помещаются куски,
        которые необходимо запросить повторно.
    \item[Протокол]--- правила, на основе которых обрабатываются передаваемые и
        получаемые сообщения.
    \item[Передача]--- некоторый файл, который передается на данный момент или как
        некоторый процесс обмена данными между участниками.
    \item[Демон]--- процесс, работающий в фоновом режиме без прямого взаимодействия с
        пользователем.
    \item[Поток] (англ. thread)~--- наименьшая единица обработки, исполнение которой может быть назначено ядром операционной системы.
    \item[Сериализация]--- процесс перевода какой-либо структуры данных в последовательность битов.
    \item[Десериализация]--- процесс, обратный процессу сериазации.
    \item[Callback-функция] (англ., функция обратного вызова)~--- передача исполняемого кода в качестве одного из параметров
        другого кода.
\end{description}
