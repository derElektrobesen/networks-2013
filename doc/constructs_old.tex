\section{Конструкторский раздел}
В данном разделе на основании разработанной в аналитическом разделе
ER\,--\,модели опишем таблицы базы данных, определим необходимые ограничения для каждой
таблицы, а так же приведем описание дополнительных средств, используемых для обработки
данных, такие как запросы, хранимые процедуры, представления и тп.

\subsection{Описание таблиц базы данных}
Колонка в таблице представляет некоторый атрибут. Для каждой таблицы необходимо определить
некоторый первичный ключ, который должен быть уникальным в пределах текущей таблицы, а так
же определить некоторые ограничения, вроде уникальности некоторых полей. При этом, при
связи <<Один ко многим>> на стороне \textit{n} создается внешний ключ, который ссылается
на родительский ключ стороны \textit{1}.
Необходимо соблюдать ссылочную целостность данных во избежание недоразумений. Далее будем
представлять логическую структуру таблицы, так же оговорим тип каждого атрибута таблицы.
Сценарий для создания SQL таблиц содержится в приложении \ref{tables}.
